%\pdfminorversion=4
\documentclass{beamer}
%\documentclass[notes]{beamer}       % print frame + notes
%\documentclass[notes=only]{beamer}   % only notes


%\setbeameroption{show notes on second screen=right}\nofiles{}

\iffalse
	\usepackage{pgfpages}
	\setbeameroption{show notes}
	\setbeameroption{show notes on second screen=right}
	% pdfpc slajdovi.pdf --notes=right
\fi

\usepackage[scaled]{beramono}				% sans-serif monospace
\input{headers/figures}
\input{headers/diagrams}
\usepackage[croatian]{babel}		% teorem

\usepackage{amsmath}

\usepackage{commath}  % abs, norm, derivacije (\od[2]{f}{x}, \od[2]{f}{x} \dif x = dx)


\usepackage{bm}

%\newtheorem{definition}{Definicija}[section]
%\newtheorem{theorem}{Teorem}[section]
%\newtheorem{corollary}{Korolar}[theorem]

\DeclareMathOperator{\step}{step}
\DeclareMathOperator{\Heaviside}{H}
\DeclareMathOperator{\Ramp}{R}
\DeclareMathOperator{\softplus}{softplus}
\DeclareMathOperator{\softmax}{softmax}
\DeclareMathOperator{\logistic}{\sigma}

\newcommand{\mat}[1]{\bm{#1}}
%\newcommand{\m}[1]{\bm{#1}}
\let\vec\relax
\newcommand{\vec}[1]{\bm{#1}}
%\newcommand{\v}[1]{\bm{#1}}
\newcommand{\tens}[1]{\bm{\mathsf{#1}}} 	% undergraduate algebra version

\newcommand{\transpose}{\mathsf T} 	% undergraduate algebra version

\newcommand{\pderiv}[2]{\frac{\partial #1}{\partial #2}}
\newcommand{\deriv}[2]{\frac{\partial #1}{\partial #2}}  % TODO

\DeclareMathOperator*{\argmin}{arg\,min} % thin space, limits underneath in displays
\DeclareMathOperator*{\argmax}{arg\,max} % thin space, limits underneath in displays

\DeclareMathOperator{\sgn}{sgn}

\usepackage{stmaryrd}  % llbracket \rrbracket, ...


%\usepackage{tabularx}
\usepackage{multirow}

\usepackage[multiple]{footmisc}		% višestruke fusnote

\usepackage{dirtree}

\usepackage[hidelinks]{hyperref}

\usepackage{xcolor}
\hypersetup{
	colorlinks,
	linkcolor={blue!50!green!50!black},
	citecolor={green!40!black},
	urlcolor={blue!75!green!30!black}
}

\usepackage[]{algorithmic}

\usepackage{color}
\definecolor{bluekeywords}{rgb}{0.13,0.13,1}
\definecolor{greencomments}{rgb}{0,0.5,0}
\definecolor{redstrings}{rgb}{0.9,0,0}
\usepackage{tabularx}
\usepackage{multirow}

\newcommand{\ilustracija}[1]{\input{ilustracije/#1}}

\iftrue
	\usepackage[croatian]{babel}
	\usepackage[utf8x]{inputenc}	
\fi

\mode<presentation>
{
	\usetheme{Boadilla}      % or try Darmstadt, Madrid, Warsaw, ...
	\usecolortheme{orchid} % or try albatross, beaver, crane, ...
	\usefonttheme{structurebold}  % or try default, serif, structurebold, ...
	\setbeamertemplate{navigation symbols}{}
	\setbeamertemplate{caption}[numbered]
}
\setbeamercolor{structure}{fg=blue!75!green!80!black}	

%\setbeamertemplate{itemize items}[default]
%\setbeamertemplate{enumerate items}[default]
\setbeamertemplate{section in toc}[circle]
\setbeamertemplate{subsection in toc}[circle]
\setbeamertemplate{items}[circle]
\setbeamertemplate{blocks}[default]
\setbeamertemplate{footline}
{
	\leavevmode%
	\hbox{%
		\begin{beamercolorbox}[wd=.15\paperwidth,ht=2.25ex,dp=1ex,center]{author in head/foot}%
			%\usebeamerfont{author in head/foot}\insertshortauthor
		\end{beamercolorbox}%
		\begin{beamercolorbox}[wd=.7\paperwidth,ht=2.25ex,dp=1ex,center]{author in head/foot}%
			\usebeamerfont{title in head/foot}\insertshorttitle  %\hspace*{3em}
		\end{beamercolorbox}%
		\begin{beamercolorbox}[wd=.15\paperwidth,ht=2.25ex,dp=1ex,right]{author in head/foot}%
			\insertframenumber{} / \inserttotalframenumber\hspace*{1ex}
		\end{beamercolorbox}
	}%
	\vskip0pt%
}

\AtBeginSection[]
{
%	\begin{frame}<beamer>
%		\frametitle{Sadržaj}
%		\tableofcontents[currentsection]
%	\end{frame}
}


\title{Parsevalove mreže}
\author{Ivan Grubišić \\ \emph{Voditelj:} Siniša Šegvić}
%\author{Voditelj: Siniša Šegvić}
\institute{Fakultet elektrotehnike i računarstva}
\date{}


\begin{document}
	
\begin{frame}
	\titlepage
\end{frame}


\begin{frame}{Sadržaj}
  \tableofcontents
\end{frame}
\note[itemize]{}

\section{Rizik kod nadziranog učenja}

\begin{frame}{Rizik kod nadziranog učenja}	
	\begin{itemize}
		\item Cilj algoritma nadziranog strojnog učenja je po parametrima modela $\theta$ minimizirati rizik $R(\theta)$ nad razdiobom označenih primjera $\mathcal{D}$. Uz odabir odgovarajućeg gubitka $L$, rizik se ovako definira:
		\begin{equation}
		R(\theta) = \mathbb{E}_{(x,y)\sim\mathcal{D}}\left[L(x, y; \theta)\right].
		\end{equation}
		\item Moguće je minimizirati procjenu rizika na temelju dostupnih podataka -- empirijski rizik.
	\end{itemize}
\end{frame}
\note[itemize]{}

\section{Neprijateljski primjeri}

\begin{frame}{Neprijateljski primjeri}	
\begin{itemize}
	\item I za najbolje klasifikacijske modele moguće je pronaći primjere jako slične prirodnima, ali da ih model potpuno krivo klasificira.
	\item Na slici je prikazano generiranje neprijateljskog primjera malom izmjenom izvorne slike.	
	\begin{figure}[htbp]
		\centering
		\begin{tabular}{>{\centering\arraybackslash}m{.2\textwidth}m{.5in}>{\centering\arraybackslash}m{.2\textwidth}m{.1in}>{\centering\arraybackslash}m{.2\textwidth}}
			\centering\arraybackslash
			%abs max for panda was 138, eps was 1., so relative eps is ~.007
			\includegraphics[width=.2\textwidth]{panda_577.png} &%
			\centering\arraybackslash%
			$\ +\ .007\ \times$ &%
			\includegraphics[width=.2\textwidth]{nematode_082.png} &%
			$=$ & %
			\includegraphics[width=.2\textwidth]{gibbon_993.png} \\
			$\centering x$     &%
			& $\sgn (\nabla_x L(x, y; \theta) )$ & & $x'$ \\
			"panda" (0.577) & & & & "gibon" (0.993) 
		\end{tabular}
		\caption{Prilagođeni prikaz dobivanja neprijateljskog primjera FGSM-om iz (goodfellow14-ehae) Riječi pod navodnicima predstavljaju rezrede, a brojevi u zagradama vjerojatnosti koje mreža dodjeljuje razredima.}
		\label{panda}
	\end{figure}	
\end{itemize}
\end{frame}
\note[itemize]{}

\begin{frame}{Pronalaženje neprijateljskih primjera}
	\begin{itemize}
		\item Neka $ B_\epsilon(x)$ označava skup primjera takvih da je njihova udaljenost od prirodnog primjera $x$ manja od $\epsilon$. 
		\item Neprijateljski primjeri se pronalaze rješavanjem optimizacijskog problema s ograničenjem:
		\begin{equation}
		x' = \argmax_{x'\in B_\epsilon(x)} L(x', y; \theta).
		\end{equation}
		\item Ako su poznati parametri mreže koju se napada, neprijateljske primjere moguće je pronaći postupcima koji se temelje na gradijentnom spustu.
		\item Mogući su i napadi crne kutije -- kada nisu poznati parametri ili struktura mreže, npr. genetskim algoritmom.
		\item Također, pokazalo se da su neprijateljski primjeri u velikoj mjeri prenosivi između različitih modela.
	\end{itemize}
\end{frame}

\begin{frame}{Pronalaženje neprijateljskih primjera}
\begin{itemize}
	\item Već je jednim pomakom u smjeru predznaka gradijenta moguće pronalaziti neprijateljske primjere (\emph{fast gradient sign method}-FGSM):
	\begin{equation}
	x' = x + \epsilon\sgn \nabla_x L(x, y; \theta).
	\end{equation}
	\item Jači su iterativni postupci kao što je PGD (\emph{projected gradient descent}):
	\begin{equation} \label{eq:pgd}
	x_{i+1} = \Pi_{B_\epsilon(x)} (x_i + \alpha\sgn \nabla_x L(x, y; \theta)).
	\end{equation}
\end{itemize}
\end{frame}



\begin{frame}{Neprijateljski rizik}
\begin{itemize}
	\item Može se definirati oblik rizika koji se može nazvati \emph{neprijateljskim rizikom}:
	\begin{equation}\label{eq:adv-risk}
	R'(\theta) = \mathbb{E}_{(x,y)\sim\mathcal{D}}\left[
	\max_{x' \in B_\epsilon(x)} L(x, y; \theta) \right].
	\end{equation}
	\item Mali neprijateljski rizik predstavlja dobru lokalnu generalizaciju u susjedstvu prirodnih primjera.
\end{itemize}
\end{frame}

\begin{frame}{Učenje s neprijateljskim primjerima}
\begin{itemize}
	\item Trenutno najuspješniji pristup za postizanje otpornosti na neprijateljske primjere je učenje s neprijateljskim primjerima (engl. \emph{adversarial training}).
	\item Kod učenja s neprijateljskim primjerima skup za učenje se proširuje neprijateljskim primjerima koji se tijekom učenja prilagođavaju parametrima mreže.
\end{itemize}
\end{frame}

\section{Parsevalove mreže}

\begin{frame}{Parsevalove mreže}
\begin{itemize}
	\item Kod Parsevalovih mreža se kontrolira Lipschitzova konstanta svih slojeva i cijele mreže tako da ne bude veća od 1.
	\item Motivacija je postizanje otpornosti na neprijateljske primjere kod dubokih neuronskih mreža.
	\item Prema autorima, takve mreže postižu bolju otpornost na naprijateljske primjere generirane FGSM-om od odgovarajućih mreža koje nisu Parsevalove, brže se uče i njihov kapacitet se bolje iskorištava.
\end{itemize}
\end{frame}

\begin{frame}{Parsevalove mreže: Ograničavanje neprijateljskog rizika Lipschitzovom konstantom}
\begin{itemize}
	\item Neka je $z(x)$ funkcija koju predstavlja sloj logita s obzirom na ulaz mreže (izlaz je $h(x)=\softmax(z(x))$).
	\item Gubitak unakrsne entropije je $L(h(x;\theta), y) = -\ln h(x;\theta)_y$.
	\item Gubitak izražen preko $z$: $\ell(z(x;\theta),y) := L(h(x;\theta), y) = -z(x;\theta)_y + \ln \sum_{y'\in\mathcal{Y}}\exp(z(x)_{y'}).$
	\item Neka za zadanu $p$-normu postoji $\lambda_p$ takav da
	\begin{equation}\label{eq:loss-lipschitz}
	\forall z,z'\in \mathbb{R}^C, \forall y\in\mathcal{Y},
	\lvert \ell(z,y)-\ell(z',y)\rvert 
	\leq \lambda_p\lVert z-z'\rVert_p.
	\end{equation}
\end{itemize}
\end{frame}

\begin{frame}{Parsevalove mreže: Ograničavanje neprijateljskog rizika Lipschitzovom konstantom}
\begin{itemize}
	\item Za svaki $p$ i $\epsilon > 0$  iz izraza~\ref{eq:loss-lipschitz} i definicije rizika $R(\theta)$ i neprijateljskog rizika $R'(\theta)=R'(\theta,p,\epsilon)$ može se pokazati da vrijedi
	\begin{align}
	R'(\theta) 
	&\leq R(\theta) + \mathbb{E}_{(x,y)\sim\mathcal{D}}\left[
	\max_{x' \in B_\epsilon(x)} \lvert \ell(z(x;\theta),y)-\ell(z(x;\theta),y) \rvert \right] \\
	&\leq R(\theta) + \lambda_p\Lambda_p\epsilon.
	\end{align}
	\item Budući da uvijek vrijedi $R(\theta) \leq R'(\theta)$, slijedi
	\begin{equation}
	0 \leq R'(\theta)-R(\theta) \leq \lambda_p\Lambda_p\epsilon.
	\end{equation}
	\item Smanjivanje Lipschitzove konstante samo po sebi nije dovoljno za poboljšanje otpornosti na neprijateljske primjere bez da se našteti općoj generalizaciji.
	\item Npr. skaliranje logita nekom malom konstantom prje $\softmax$-a smanjuje $R'(\theta)-R(\theta)$, ali ne utječe na otpornost.
\end{itemize}
\end{frame}

\begin{frame}{Literatura}
\bibliography{literatura}
\bibliographystyle{fer}

\nocite{colah}
\nocite{nndl}

\end{frame}
\end{document}
